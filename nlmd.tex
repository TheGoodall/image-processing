\documentclass[11pt]{article}
\usepackage[margin=0.5in]{geometry}
\usepackage{cite}
\title{Software methodologies: Image processing: A report on Non-Local Means Denoising}
\author{James Goodall}

\begin{document}
\maketitle
\section{The non-local means denoising algorithm.}

Non-local means is an algorithm for denoising images, based on the principle of replacing a pixel of an mean of all the pixels in the image weighted by how similar their surroundings are to the surroundings of the original pixel. 

As the introduction of \cite{anlafid} states, the goal of image denoising methods is to recover the original image from a noisy measurement.

As described in \cite{dip20}, the value of a pixel can be thought of as the sum of the original value plus a random noise element e.g.

\[P = P_0 + N\]

because of this, we can take multiple similar areas in the image we are trying to denoise, each with a different noise added to it but the same original value e.g.

\[P_1 = P_0 + N_1\]
\[P_2 = P_0 + N_2\]
\[\vdots\]
\[P_n = P_0 + N_n\]

finding the mean of $P_n$ results in the sum of $P_0$ and the average of $N$. Since $N$ can be modeled with a mean of $0$, for large values of $n$ the mean of $P_n$ tends towards $P_0$

\subsection{The Algorithm} \label{algorithm}

the algorithm is defined by \cite{Buades_2005} to be: 

\[NL(v)(i) = \sum_{j \in I}{w(i,j)v(j)}\]

with $w(i,j)$ being the similarity function, which is the square of the euclidian distance between the two areas surrounding the pixels i and j, calculated by:

\[\left\|v\left(\mathcal{N}_{i}\right)-v\left(\mathcal{N}_{j}\right)\right\|_{2, a}^{2}\]

with $\mathcal{N}_i$ refering to the pixels surrounding i

A useful property of this similarity function is that, as explained in \cite{Buades_2005} is that the Euckudean distance preserves the order of similarity between pixels, which is to say, that the similarity of $a$ to $b$ is the same as the similarity of $b$ to $a$.

\section{Implementations of the algorithm and their efficiency.}

There are two main implementations of the algorithm, pixelwise and patchwise:

\subsection{Pixelwise}

The pixelwise implemtation is as described in section \ref{algorithm}, howevere due to computational limitations, in \cite{nlmd}, the search windows, instead of being the entire image, is limited to a $21\times 21$ square around the pixel in question for small values of $\sigma$ and to a $35 \times 35$ square for large values.

\subsection{Patchwise}



\section{The influence of the algorithmic parameters on the output.}



\section{The strengths and limitations of non-local means compared to other denoising algorithms. }




\section{Modifications and extensions of the algorithm that have been proposed in the literature. }



\section{Applications of the original algorithm and its extensions. }



\bibliography{research}
\bibliographystyle{plain}
\pagebreak
\section{appendix}


\end{document}
